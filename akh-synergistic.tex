%!TEX root = desc.tex
\newif\ifdraft\drafttrue

\documentclass[11pt]{article}
\usepackage[margin=1in]{geometry}
\usepackage[sort&compress,numbers]{natbib}
% amsthm needs to go before newtx for compatability reasons
\usepackage{amsthm,amssymb}
\usepackage{newtx}
\usepackage{microtype}
\frenchspacing

\usepackage{mathpartir}
\usepackage{stmaryrd}
\usepackage{mathtools}

\usepackage{suffix}
\usepackage{xparse}

%\usepackage{pslatex}
%\usepackage{fullpage}
\usepackage{xspace}
\usepackage{verbatim}
\usepackage{enumitem}
%\usepackage{wrapfig}
%\usepackage{tgpagella}
\usepackage{url}
\usepackage{breakurl}
\usepackage[hidelinks,bookmarks=true]{hyperref}
\usepackage{graphicx}
\usepackage{subfig}
\usepackage{titlesec}
%\usepackage{fancyhdr}
\usepackage{float}
\usepackage[font=small]{caption}
\usepackage[table,dvipsnames]{xcolor}

\setlist[itemize,enumerate]{
  itemsep=0pt,
  topsep=1ex,
}

% ===== Title and section header formatting ====
\makeatletter
\def\@author{}
\def\author#1#2{\expandafter\def\expandafter\@author\expandafter{\@author{#1}{#2}}}
\def\typeset@authors#1#2#3\_nil{%
  \begin{tabular}[t]{c}#1\end{tabular}\\
  {\large #2}
  \ifx\relax#3\relax\else\and\typeset@authors#3\_nil\fi%
}
\def\maketitle{
  \newpage%
  \vbox to 1.2in{
    \centering
    \vspace*{\fill}
      {\LARGE\bfseries\@title\par}
      \vskip 0.75em
      {\Large\begin{tabular}[t]{c}\expandafter\typeset@authors\@author\_nil\end{tabular}\par}
    \vspace*{\fill}
  }
%  \begin{titlepage}
%    \centering
%    \vspace*{\fill}
%      {\huge\@title\par}
%      \vskip 2em
%      {\Large\begin{tabular}[t]{c}\expandafter\typeset@authors\@author\_nil\end{tabular}\par}
%    \vspace*{\fill}
%  \end{titlepage}
}
%\def\typeset@affils#1#2#3\_nil{\textsuperscript{#1}#2\ifx\relax#3\relax\else\and\typeset@affils#3\_nil\fi}
%\newcommand{\affils}[1]{\def\@affils{#1}}
%\def\maketitle{
%  \newpage%
%  \vbox to 0.9in{
%    \centering
%    \vspace*{\fill}
%      {\Large\bfseries\@title\par}
%      \vskip 1ex
%      {\normalsize\begin{tabular}[t]{c}\@author\end{tabular}\par}
%      \ifx\relax\@affils\relax\else
%        \vskip 0.5ex
%        {\small\begin{tabular}[t]{c}\expandafter\typeset@affils\@affils\_nil\end{tabular}\par}
%      \fi
%    \vspace*{\fill}
%  }
%}
\makeatother

\titlespacing*{\section}{0pt}{*2.5}{*2}
\titlespacing*{\subsection}{0pt}{*2}{*1}
%\titlespacing*{\subsubsection}{0pt}{*0.75}{*0.5}
\titleformat{\paragraph}[runin]{\bfseries}{}{0pt}{}[.]
\titlespacing{\paragraph}{0pt}{*1.5}{*1}
\titleformat{\subparagraph}[runin]{\bfseries}{}{0pt}{}
\titlespacing{\subparagraph}{0pt}{*1}{*1}

\makeatletter
\newcounter{ex@cntr}
\newenvironment{example}[1]{%
  \par\vspace{1.5ex plus 0.5ex minus 0.5ex}%
  \noindent%
  \addtocounter{ex@cntr}{1}%
  \textbf{Example \theex@cntr: #1.}\hspace{1ex plus 0.5ex minus 0.5ex}\ignorespaces%
}{\par\medskip}
\makeatother

%\renewcommand{\textfraction}{0.01}
%\setcounter{topnumber}{100}
%\setcounter{dbltopnumber}{100}
%\setcounter{totalnumber}{100}
%\renewcommand{\topfraction}{.9}
%\renewcommand{\floatpagefraction}{.9}
%\renewcommand{\dbltopfraction}{.9}
%\renewcommand{\dblfloatpagefraction}{.9}
%
%\newcounter{cnt}

%\setlength{\abovecaptionskip}{0in}
%\setlength{\belowcaptionskip}{0in}


%make latex layout figures better
%\renewcommand{\topfraction}{0.85}
%\renewcommand{\textfraction}{0.09}
%\renewcommand{\floatpagefraction}{0.85}


\date{}

%\newsavebox{\mylistingbox}
%
%\pagestyle{fancy}
%\fancyhead{}
%\fancyhf{}
%\renewcommand{\headrulewidth}{0pt}
%\setcounter{page}{1}
%%\usepackage{epstopdf}

% Notes
\newcommand{\newcommenter}[3]{%
  \ifdraft
    \newcommand{#1}[1]{%
      \begingroup\small\sffamily\color{#2}%
        [#3: ##1]%
      \endgroup%
    }%
  \else
    \newcommand{#1}[1]{\ignorespaces}
  \fi
}
\definecolor{darkgreen}{rgb}{0,0.7,0}

\newcommenter{\ethan}{darkgreen}{Ethan}
\newcommenter{\akh}{purple}{AKH}

\newcommand{\todo}[1]{\bgroup\small\sffamily\color{red}[\textbf{Todo:} #1]\egroup}

%% === Program Syntax ===
\newcounter{numlevels}
\NewDocumentCommand{\newMLP}{sO{l}mO{0}om}{
  \IfBooleanTF{#1}{
    \IfValueTF{#5}{
      \WithSuffix\newcommand#3*[#4][#5]
    }{
      \WithSuffix\newcommand#3*[#4]
    }
  }{
    \IfValueTF{#5}{
      \newcommand{#3}[#4][#5]
    }{
      \newcommand{#3}[#4]
    }
  }
  {\ifthenelse{\value{numlevels} > 0}{\begin{array}[t]{@{}#2@{}}}{\begin{array}{#2}}\addtocounter{numlevels}{1}#6\addtocounter{numlevels}{-1}\end{array}}
}
\makeatletter
\def\def@block@inner#1#2#3\_nil{#1 & #2\ifx\relax#3\relax\else \\ \def@block@inner#3\_nil\fi}%
\newcommand{\typeset@def@block}[2]{\begin{array}[t]{@{}l@{{}#1{}}l@{}}\def@block@inner#2\_nil\end{array}}
\makeatother

\newcommand{\ty}{\mkern2mu{:}\mkern2mu}
\newcommand{\proves}{\vdash}
\newcommand{\nproves}{\nvdash}
\newcommand{\dom}[1]{\operatorname{dom}({#1})}
\newcommand{\seq}{\mathrel{;}}
\newcommand{\sendsto}{\rightsquigarrow}

\definecolor{chorcolor}{HTML}{0053A9}
\definecolor{localcolor}{HTML}{007C21}
\definecolor{ntwkcolor}{HTML}{E44223}
\definecolor{choicecolor}{HTML}{BB7A00}
\definecolor{loccolor}{HTML}{b51963}
\newcommand{\ChorCol}[1]{{\color{chorcolor}#1}}
\newcommand{\LocalCol}[1]{{\color{localcolor}#1}}

\newcommand{\programfont}[1]{\ensuremath{\mathsf{#1}}\xspace}
\newcommand{\FontChoreo}[1]{\programfont{\ChorCol{#1}}}
\newcommand{\LocalLangFont}[1]{\programfont{\LocalCol{#1}}}
\newcommand{\LocFont}[1]{\programfont{\color{loccolor}#1}}

\newcommand{\Mngr}{\LocFont{M}}
\newcommand{\Alice}{\LocFont{A}}
\newcommand{\Bob}{\LocFont{B}}
\newcommand{\Client}{\LocFont{C}}

\newcommand{\ChorDef}{\mathrel{\ChorCol{\coloneqq}}}
\newcommand{\ColSend}{\mathrel{\ChorCol{\sendsto}}}

\newcommand{\IfN}{\FontChoreo{if}}
\newcommand{\Then}{\FontChoreo{then}}
\newcommand{\Else}{\FontChoreo{else}}
\newcommand{\ForkN}{\FontChoreo{fork}}
\newcommand{\LetN}{\FontChoreo{let}}
\newcommand{\In}{\FontChoreo{in}}
\newcommand{\FunN}{\FontChoreo{fun}}
\newcommand{\LamN}{\ChorCol{\lambda}}
\newcommand{\TLamN}{\ChorCol{\Lambda}}

\newcommand{\ForkIn}[3]{{#1}.\ForkN~{#2}~\In~{#3}}
\newMLP*{\ForkIn}[3]{{#1}.\ForkN~{#2} \\ \In~{#3}}
\newcommand{\LetIn}[3]{\LetN~{#1} \ChorDef {#2}~\In~{#3}}
\newMLP*[r@{~}l]{\LetIn}[3]{\LetN & {#1} \ChorDef {#2} \\ \In & {#3}}
\makeatletter
\newMLP{\LetMany}[2]{\LetN ~ {\typeset@def@block{\ChorDef}{#1}} \\ \In ~ {#2}}
\makeatother
\newcommand{\Fun}[3]{\operatorname{\FunN}{#1}({#2}) \ChorDef {#3}}
\newMLP*{\Fun}[3]{\operatorname{\FunN}{#1}({#2}) \ChorDef \\ {#3}}
\newcommand{\ITE}[3]{\IfN~{#1}~\Then~{#2}~\Else~{#3}}
\newMLP*{\ITE}[3]{\IfN~{#1} \\ \Then~{#2} \\ \Else~{#3}}

% === Outline structure ===
\usepackage{boxedminipage}
\makeatletter
\newcounter{outline@templateDepth}
\newcounter{outline@depth}
\newenvironment{outline}[2][0]{%
  \newlist{lvl}{enumerate}{6}%
  \setlist[lvl]{parsep=0.33\baselineskip,itemsep=0pt,topsep=0pt,leftmargin=1.5em}%
  \setlist[lvl,1]{leftmargin=*,topsep=0.5em}%
  %
  \setcounter{outline@templateDepth}{#1}%
  \setcounter{outline@depth}{1}%
  \@whilenum\value{outline@templateDepth}<6\do{%
    \ifcase\value{outline@templateDepth}\relax
      \setlist*[lvl,\value{outline@depth}]{label=\Roman*.}%
    \or
      \setlist*[lvl,\value{outline@depth}]{label=\Alph*.}%
    \or
      \setlist*[lvl,\value{outline@depth}]{label=\arabic*.}%
    \or
      \setlist*[lvl,\value{outline@depth}]{label=\alph*.}%
    \or
      \setlist*[lvl,\value{outline@depth}]{label=\roman*.}%
    \or
      \setlist*[lvl,\value{outline@depth}]{label=-}%
    \fi
    \stepcounter{outline@depth}%
    \stepcounter{outline@templateDepth}%
  }
  \if\relax\detokenize{#2}\relax\else
    \par\noindent{\bfseries #2}\par%
  \fi
  \noindent%
  \begin{lvl}%
}{\end{lvl}\vspace{0.25em}}
\newenvironment{boxedoutline}[2][0]{
  \begin{boxedminipage}{\textwidth}
  \begin{outline}[#1]{#2}
}{\end{outline}\end{boxedminipage}\vspace{0.5em}\par}
\makeatother

%%% Local Variables:
%%% mode: latex
%%% TeX-master: "desc"
%%% End:


\begin{document}

\begin{center}
{\LARGE
\textsc{Synergistic Activities for PI Andrew K. Hirsch}
}
\end{center}
\hrule
\vspace{3mm}

\begin{enumerate}
\item PI~Hirsch participates in running the Programming Languages Mentoring Workshop (PLMW) at the prestigious Principles of Programming  Languages (POPL) conference.
  This brings early graduate students and late undergraduate students to POPL and to attend a day-long workshop with lectures on soft skills, hard science, and time to make connections with both each other and leading scientists in the field.
  In particular, while the workshop is open to anyone, we provide scholarships for such students, focusing on those who have a harder time finding their way into the scientific community.
  At the workshop, the hard-science lectures provide a basis of knowledge so that students can understand talks in areas which their schools may not focus, and the soft-skills talks focus on helping students get the most out of the workshop.
\item PI~Hirsch participates in the Experiential Learning and Research~(ELR) program at the University at Buffalo, the State University of New York.
  In this program, PI~Hirsch leads a team of three-to-six undergraduates in a long-term research and engineering project.
  The focus is on developing the students' skills, rather than on developing anything that is groundbreaking.
  In this way, PI~Hirsch has trained students on hard-to-access skills like functional programming, compiler engineering, concurrent and distributed programming, and verification.
\item Every year, PI~Hirsch volunteers his time to do outreach with the Electronics Frontiers of Georgia and the Science Track at Dragoncon.
  Dragoncon is one of the country's largest science fiction conventions.
  As part of his volunteer work, PI~Hirsch helps explain scientific concepts, especially those underlying the most-common security flaws in software today.
  This allows the public access to hear from, and ask questions to, a scientist working in the field.
  These include technical questions, but also career advice, questions of the public interest, and more.
\item As a graduate student PI~Hirsch worked with the Ithaca School District to develop programming for elementary-school students interested in computer science.
  In this program, PI~Hirsch helped students understand important concepts from computer science, such as following a list of instructions, conditionals, and loops.
  All of this was done via the power of play: the students (and the instructors) would take turns acting as a robot, while the other students wrote instructions (in the form of cards provided by the instructor) that the robot needed to follow in order to accomplish a task.
  This allowed students to associate programming with a fun and collaborative game, rather than with sitting in front of the computer wondering why it did not understand.
\item Also as a graduate student, PI~Hirsch worked with the Cornell Department of Mathematics and the Ithaca School District to provide hands-on in-classroom mathematical encounters.
  Activities included building platonic solids and investigating their properties, discovering basic statistics by rolling dice, and discussing geometry through the question ``who is bigger, you or your shadow?''
  For many students (and many teachers), it was their first pleasant interaction with mathematics, and teachers commented that it made both them and their students more interested in math class.
\end{enumerate}
\end{document}

%%% Local Variables:
%%% mode: LaTeX
%%% TeX-master: t
%%% End:
