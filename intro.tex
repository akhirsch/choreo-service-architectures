%!TEX root = desc.tex
\maketitle

\section{Introduction}
\label{sec:intro}

\begin{outline}{Introduction Outline}
  \item Programming concurrent and distributed systems correctly is hard, but choreographies can help!
    \begin{lvl}
      \item A major promise of choreographies is that the implementation of a protocol
        can directly mirror the structure of a protocol specification.
      \item This idea simplifies system building, helps with rapid prototyping, and simplifies checking that the implementation matches the spec.
    \end{lvl}

    \item State-of-the-art choreographies are missing critical features needed to accomplish this goal.
    \begin{lvl}
      \item Common systems with complex protocols include
        microservice architectures, thread pools, web servers, and distributed consensus protocols (e.g., Paxos).
        \ethan{I'm not sure we want to talk about Paxos if we're not doing asynchrony.}
      \item None of these can be implemented with current choreographic technology for a fundamental reason:
        every message must be sent and received in a pre-determined order,
        and all parties must know each other's identities \emph{in advance}.
      \item The above protocols violate this assumption in multiple ways.
        \begin{lvl}
          \item They wait for messages that may arrive at any time and execute in response.
          \item Messages may come from any client, and the recipient only learns the client's identity with the message arrives.
          \item Nodes/threads/processes/etc are often spawned dynamically in response to requests
            with identities that cannot be known in advance.
            \ethan{We've partially solved this one already!}
        \end{lvl}
    \end{lvl}

  \item We propose \emph{choreographic service architectures} to support these features.
    \begin{lvl}
      \item
    \end{lvl}
  \ethan{I need to finish this later.}
\end{outline}
\hrule
\vspace{0.5em}

\paragraph{Intellectual Merit}


\subsection{Motivation ...}
\label{sec:motivation}


\subsection{Research Team}

\subparagraph{Ethan Cecchetti} is an expert...

\subparagraph{Andrew K. Hirsch} is an expert in choreographic programming...

%%% Local Variables:
%%% mode: latex
%%% TeX-master: "desc"
%%% End:
