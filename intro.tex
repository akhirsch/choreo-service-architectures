%!TEX root = desc.tex
\maketitle

\section{Project Description}
\label{sec:intro}

Nearly every major computer system now relies on substantial concurrency for performance and scalability,
yet designing and implementing correct concurrent systems is famously challenging.
The new paradigm of \emph{choreographic programming}~\citep{Montesi13,Montesi23} promises to simplify development and analysis of such systems.
Instead of writing separate programs for each system component---forcing developers to analyze the interplay of these programs---% DO NOT DELETE, COMMENT IS IMPORTANT FOR SPACING
choreographies mirror more informal protocol specifications with a single top-level program
describing both the local operations of each component---which we call \emph{processes}, though they may be threads, processes, or even entire nodes---and how they interact.

The promise of choreographic programming is twofold.
First, the implementation of a protocol can directly mirror its specification.
For instance, a protocol that includes ``Alice sends message~$m$ to Bob, who forwards it to Cathy''
could be written as the choreography $(\text{Alice}.m \sendsto \text{Bob}) \sendsto \text{Cathy}$.
A compiler can then automatically generate three separate pieces of code, one for each of Alice, Bob, and Cathy,
with separate send and receive instructions.
Contrast this with traditional programming techniques, which require developers to write the three separate pieces of code directly.
Second, the existence of a global program makes global control flows more visible.
This structure not only provides deadlock freedom by construction~\cite{CarboneM13}, but also drastically simplifies other program analyses.
In this way, choreographies aim to ease system building, help with rapid prototyping,
and simplify both checking that implementations matches specifications
and analyzing important system properties.

Rapid progress in choreographic theory in the past few years~\citep[e.g.,][]{HirschG22,CruzFilipeGLMP22,CruzFilipeGLMP23,BatesK+25,SamuelsonHC25}
has brought this promise closer into view, but substantial work remains.
In particular, current choreographic theory relies on three extremely strong assumptions
that each rule out many common architectures, including thread pools, consensus protocols, and client-server models like web servers or microservice architectures:
\begin{enumerate}
  \item\label{li:assume:static-locs}
    There must be a pre-determined set of processes.
  \item\label{li:assume:known-order}
    Those processes must send and receive messages in a pre-determined order.
  \item\label{li:assume:known-endpoints}
    The senders and recipients of each message must be known to all relevant parties in advance.
\end{enumerate}

A recent result by both PIs showing how to safely enable first-class process polymorphism~\citep{SamuelsonHC25} opens critical new avenues for research alleviating those assumptions.
Choreographies can now dynamically compute and communicate which process(es) will perform any given action,
allowing run-time flexibility and increased code reuse.
On its own, this result is an important addition to the choreographic toolbox,
but its real power lies in creating a foundation to accelerate advances in choreographic technology.

\emph{We propose to leverage process polymorphism to develop substantial new theory and implementation techniques, bringing the choreographic programming paradigm closer to its promise.}
Guiding this work will be the goal of enabling choreographies to specify common concurrent and distributed systems.
This is not possible currently, as many systems violate the assumptions above.
Processing requests often involves spawning new threads, violating assumption~(\ref{li:assume:static-locs}).
Microserves must wait for requests without knowing when or if they will arrive, violating assumption~(\ref{li:assume:known-order}).
Finally, consensus nodes simultaneously wait for messages from many senders
and only learn who sent valid messages after they arrive,
violating assumption~(\ref{li:assume:known-endpoints}).

Notably, the PIs have a longstanding and fruitful collaboration.
They have worked together not only on process polymorphism~\cite{SamuelsonHC25}, but on information-flow control~\cite{SilverHCHZ23,HirschC21} and authorization logic~\cite{HirschACAT20}.
They have additionally developed preliminary work for this grant together, as described below; PI~Hirsch has also developed preliminary work in the form of a choreographic compiler as described below.

\subsection{Intellectual Merit}

To alleviate the need for the assumptions listed above and allow choreographic programming to achieve its promise in practical systems, we plan to pursue the following research:
\begin{itemize}
  \item \textit{\hyperref[sec:t1]{Thrust 1} (Polymorphism-Enabled Features):}
  \item \textit{\hyperref[sec:t2]{Thrust 2} (Improved Compilation for Process Polymorphism):}
  \item \textit{\hyperref[sec:t3]{Thrust 3} (Better Guarantees for Process-Polymorphic Programs):}
\end{itemize}

\ethan{I'm going to come back to this section and tighten it up once there's a draft of the trusts.
That way we can make sure it all matches up.}

In \hyperref[sec:t1]{Thrust 1}, we will develop new features enabled by process polymorphism.
In particular, we plan to develop the ability to spawn new threads dynamically in a program, allowing them to interact with other choreographic participants in a first-class way.
We also plan to develop choreographies wherein one participant can wait for any number of other participants to send a message, receiving those messages in any order.
In particular, we plan to develop a version of this feature that will allow for a programmer to specify exactly what pattern of messages a program must receive before proceeding.

In \hyperref[sec:t2]{Thrust 2}, we will examine how process polymorphism is implemented in more detail.
In particular, the current theory of process polymorphism tends to require an expensive compilation step which can result in a single process's code being exponentially bigger than the choreographic code.
We plan to optimize this process in common cases, allowing for smaller and cheaper concurrent code.
In order to validate this, we also plan to develop a benchmark of process-polymorphic programs, additionally helping speed the development of further process polymorphism technology.

Finally, in \hyperref[sec:t3]{Thrust 3}, we will explore new, more powerful, guarantees for process-polymorphic programs.
In particular, we plan to make the deadlock-freedom guarantee provided by choreographies more subtle and more resilient.
To make the guarantee more subtle, we plan to develop a version of deadlock freedom that allows for some processes to be waiting for an unknown client to contact them.
Intuitively, a state in which all programs are either done or waiting for a client is not in deadlock; instead, every program did everything it needs to do.
Current choreographic guarantees would rule this out, however.
To make the guarantee more resilient, we plan to study how we can provide deadlock-freedom guarantees in the face of nonterminating processes.
Currently, if a process goes into an infinite loop in a choreography, the choreographic semantics say that no other process takes any actions.
We plan to make a better, more resilient guarantee by turning to the mathematical theory of the ordinals, which allows us to ``see past'' the infinite loops and make guarantees about other processes.

\subsection{Research Team}

As stated above, the PIs have a longstanding and fruitful collaboration.
They have weekly remote meetings, and visit in person at least annually.
Moreover, they each bring unique skills to the table which have proven synergistic many times in the past.

\subparagraph{Ethan Cecchetti} is an expert in applying programming-language theory to systems.
He is a coinventor of first-class process polymorphism for choreographies~\citep{SamuelsonHC25}.
He has established links between foundational frameworks in programming languages and cryptography~\citep{KuennemannPC24,PatrignaniKWC24},
developed foundational theory around information-flow control~\citep{HirschACAT20,HirschC21,SilverHCHZ23},
and shown how to generalize standard language-based conditions to more nuanced contexts~\citep{CecchettiMA17,CecchettiYNM21,Cecchetti25}.
Finally, he has used applied cryptographic tools to design protocols for real-world decentralized systems~\citep{ZhangCCJS16,CecchettiZJKJS17,CrooksBCHAA18,CecchettiFMJ19}.

\subparagraph{Andrew K. Hirsch} is an expert in the theory of programming languages.
In particular, he is one of the inventors of functional choreographic programming~\cite{HirschG22} and process polymorphism~\cite{GraversenHM24}, and he has developed considerable expertise in the area~\cite{SamuelsonHC25,GraversenHM24,HirschG22,BohosianH25}.
He is also an expert in information-flow control~\cite{SilverHCHZ23,MenzHLG23,HirschC21,HirschACAT20}, the theory of effects~\cite{SilverHCHZ23,HirschC21,HirschT18}, and authorization logic~\cite{HirschACAT20,HirschC13}.

%%% Local Variables:
%%% mode: latex
%%% TeX-master: "desc"
%%% End:
