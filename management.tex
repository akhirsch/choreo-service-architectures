%!TEX root = desc.tex

\section{Project Timeline and Activities}
\label{sec:man}

The work plan for this project is given in Figure~\ref{fig:workplan} in a quarterly breakdown based on the research goals listed for each research thrust in section~\ref{sec:research}.
The goals found in \hyperref[sec:t1]{Thrust~1} are in \textcolor{cyan}{cyan}, the goals found in \hyperref[sec:t2]{Thrust~2} are in \textcolor{green}{green}, and the goal found in \hyperref[sec:t3]{Thrust~3} are in \textcolor{magenta}{magenta}.
Solid bars indicate more effort.

\begin{figure*}
  \centering
  %from https://tex.stackexchange.com/questions/629188/gantt-bars-with-multicolour-diagonal-stripes
  \pgfqkeys{/pgfgantt}{
    bar pattern/.style={
      bar/.append style={
        pattern={Stripes[angle=45,distance=5pt,color series={#1}]}
      },
    },
    bar pattern/.default={red,green,blue},
  }
  \begin{ganttchart}[hgrid=true,vgrid=true,x unit=2.5em, y unit chart=1.2em, bar height=.6, bar top shift=0.2]{1}{12}
    \gantttitlelist[title list options={var=\y, evaluate=\y as \x using "Year \y"}]{1,...,3}{4}\\
    % Repeats 3 times, once for each year
    \foreach \i in {1,...,3} {
      \gantttitlelist[title list options={var=\y, evaluate=\y as \x using "Q \y"}]{1,...,4}{1}
    }\\
    \ganttbar[bar pattern={cyan}]{\hyperref[goal:spawn]{\small Goal 1 (Spawning Threads)}}{1}{3}
    \ganttbar[bar pattern={cyan,white}]{}{4}{4}\\
    \ganttbar[bar pattern={cyan,white}]{\hyperref[goal:multirecv]{\small Goal 2 (Multireceive)}}{3}{3}
    \ganttbar[bar pattern={cyan}]{}{4}{8}
    \ganttbar[bar pattern={cyan,white}]{}{8}{8}\\
    \ganttbar[bar pattern={green,white}]{\hyperref[goal:opt-epp]{\small Goal 3 (EPP Optimization)}}{5}{5}
    \ganttbar[bar pattern={green}]{}{6}{10}
    \ganttbar[bar pattern={green,white}]{}{11}{11}\\
    \ganttbar[bar pattern={green}]{\hyperref[goal:benchmark]{\small Goal 4 (Benchmark)}}{8}{12}\\
    \ganttbar[bar pattern={magenta}]{\hyperref[goal:strong-epp]{\small Goal 5 (New Bisimulations)}}{1}{6}
    \ganttbar[bar pattern={magenta,white}]{}{7}{7}\\
    \ganttbar[bar pattern={magenta,white}]{\hyperref[goal:wait-until-called]{\small Goal 6 (Service Functions)}}{7}{7}
    \ganttbar[bar pattern={magenta}]{}{8}{12}
  \end{ganttchart}
%  \includegraphics[width=.7\linewidth]{GANTchart.png}
\caption{The proposed work plan split into quarterly tasks for the duration of the grant period.}
\label{fig:workplan}
\end{figure*}

Since our thrusts are largely independent, they begin executing in parallel.
In particular, we already have preliminary work on Goals~\ref{goal:spawn} and \ref{goal:strong-epp}, so we plan to begin the starting period of the grant by working on those.
Our preliminary work suggests that the ``radical'' work on new bisimulations (Goal~\ref{goal:strong-epp}) will take some time, which is why it is spread over six quarters.
However, we expect the choreographic-feature development of Goals~\ref{goal:spawn}, \ref{goal:multirecv}, and~\ref{goal:wait-until-called} will take approximately equal amounts of time.
We thus sequentialize them, overlapping by one quarter to represent beginning the work on one while preparing the previous for publication.

We save the implementation tasks (Goals~\ref{goal:opt-epp} and~\ref{goal:benchmark}) to the end of the grant period so that we can implement as many of the features we plan to develop as possible.
We can also start to assemble a benchmark (Goal~\ref{goal:benchmark}) almost immediately after we begin implementation.
This is seen in Figure~\ref{fig:workplan} by the bar for Goal~\ref{goal:benchmark} starting in Year 2, Quarter 4, while the bar for the main implementation goal (Goal~\ref{goal:opt-epp}) begins two quarters earlier.
The first quarter we expect to be fully devoted to the development of the theory, while the second quarter we will begin work on the implementation.

%%% Local Variables:
%%% mode: latex
%%% TeX-master: "desc"
%%% End:
