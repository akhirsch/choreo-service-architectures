%thrust3.tex
\subsection{Thrust 3 - Better Guarantees for Process-Polymorphic Programs}
\label{sec:t3}

\vspace*{0.5em}
\hrule
\vspace*{0.5em}

\ethan{Here's some text to build off of that I put into my grant insurance application}

This work aims to address this shortcoming and develop the theory required to support service architectures in choreographic programming.
In particular, it aims to address two core research questions in tandem.

(1) How can we represent services in choreographic programs?

The assumption that senders and receivers know each other's identities before messages are sent
is embedded into the structure and syntax of existing choreographic languages.
To ensure message sends and receives match up, and thus avoid deadlock,
choreographic languages do not have separate send and receive instructions,
but rather a single combined instruction that must specify the sender, recipient, and information being sent.
This structure does not support the kind of client-server service architecture described above.

We will therefore develop new language structures to specify these architectures within choreographies.
They will include sending messages, but without the structural requirements of existing languages.
Instead, the program for a server will only specify that it is waiting for a message from any client,
while clients will be allowed to send requests to servers without identifying themselves in advance.

(2) How do we formally define and enforce deadlock freedom with choreographic services?

Being able to represent services in a choreographic language is not sufficient to realize the promise of choreographic programming.
We also need to be able to analyze the behavior of the programs and continue to guarantee deadlock freedom.
However, existing formal mathematical definitions of deadlock freedom require that,
if the system cannot continue executing, then every component must have terminated.
With services, however, this is not the case.
A service waiting for a request when all potential clients have terminated is not itself terminated, but is also not deadlocked.
This work aims to resolve this discrepancy.
It will identify an appropriate definition of deadlock freedom and structure
the choreographic service representation to enforce this new, more general notion.

\vspace*{0.5em}
\hrule
\vspace*{0.5em}



%%% Local Variables:
%%% mode: latex
%%% TeX-master: "desc"
%%% End:
