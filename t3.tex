%thrust3.tex
\subsection{Thrust 3 - Choreographic Services}
\label{sec:t3}

While choreographies in the state of the art excel at writing multiprocess programs, they are not able to express \emph{services}.
Services require some \emph{provider} process(es) to wait for \emph{requests} from (an) unknown \emph{client} process(es).
Once that request is received, some code is executed to \emph{satify} that request.
Afterwords, the provider goes back to waiting for further requests.

It may seem surprising that process polymorphism does not immediately provide the ability to create services.
We can create a process-polymorphic function that allows anyone to interact with a provider.
However, in order to run such a function the provider would already have to know the identity of the client, whereas a service waits for a request from an unknown client.
Using the PI's joint work on first-class process polymorphism~\cite{SamuelsonHC25} could get around this problem, but would then require a known third party to tell the server the identity of the client.
The ability to await requests from truly unknown clients does not exist in choreographies.

Adding the ability to write services immediately runs into two problems, one small and one large.
To understand the big problem, imagine that \Alice{} and \Bob{} are both trying to make requests to a service provider~\Server.
Since the underlying communication substrate (such as a network in a distributed system) may deliver either message first, \Server{} could correctly begin satisfying either request.
Once it has begun to serve one request, it cannot start to work on the other.
This all happens despite the fact that in the choreographic program one request will certainly come before the other.
Choreographic languages must therefore have some notion of \emph{out-of-order execution}.
We believe that we can modify the notions of out-of-order execution used by previous works to fit choreographic services.
However, by taking a more-radical approach, we believe we can get stronger results and more elegant proofs.

To understand the small problem, imagine again that \Alice{} and \Bob{} are both trying to make requests to a service provider~\Server.
After both requests are satisfied, \Server{} will go back to waiting.
Intuitively this is not deadlock, despite the fact that a process waiting for a message that will never come is usually a sign of deadlock.
We plan to alleviate this problem by adding distinguished \emph{service functions} which act like waits, but do not count as deadlocks.
These service functions will likely require some restrictions to avoid deadlock themselves.

The rest of this subsection discusses the two problems above in more detail and proposes solutions.
We emphasize that solving the big problem using the radical approach we suggest here is not required.
We suggest it because leads to an elegant solution with strong guarantees; however, if it fails we still believe we will be able to provide a deadlock-free notion of choreographic services.

\subsubsection{A New Paradigm for Bisimulations}
\label{sec:new-parad-bisim}

Previous work on choreographic languages focused on proving that endpoint projection induced a \emph{bisimulation} between the choreographic code and the projected code.
A bisimulation says that any action taken by the choreographic code can be taken by the projected code and vice-versa.
Note that the projected code allows actions taken by disjoint processes to happen in any order; because of this, almost all choreographic languages allow for \emph{out-of-order execution}~\cite{Montesi23,HirschG22,Cruz-FilipeM17}.
Once a bisimulation has been established, deadlock freedom follows immediately from standard safety results of single-threaded languages.

While bisimulations allow for extremely strong results, they are nearly impossible to establish for functional choreographic languages.
This has led authors to either weaken their choreographic language considerably (which has been the PIs' approach)~\cite{HirschG22,SamuelsonHC25} or to apply fragile techniques that do not allow for basic additions to the language~\cite{CruzFilipeGLMP23}.
As more and more features get added to functional choreographic languages, developing a better technique for proving deadlock freedom will become paramount.
Moreover, such a technique will need to allow other properties of the choreographic language to be transported to its projection.

We observe that such proofs do not require a full bisimulation between programs.
Instead we can look at the possible \emph{traces} (or lists of actions taken during a particular run of a program) of both the choreography and the projected network.
As long as these are \emph{equivalent} (equal up to rearranging actions of disjoint processes), then endpoint projection preserves that semantics of the choreography.
This is enough to allow us to prove deadlock freedom and to transport other important properties from the choreographic language to its projection.
Thus, we do not need to allow out-of-order execution of the choreography.
Instead, we simply define its in-order trace-collecting execution, and then consider this trace up to equivalence.

We believe that this scheme works perfectly for terminating languages.
However, nonterminating languages allow things to get hairier.
Consider the choreography $\Alice.(\programfont{loop}\,()); \Bob.42 \ColSend \Cathy$.
In this program, \Alice{} enters an infinite loop ``and then'' \Bob{} sends 42 to \Cathy.
If we were to use the scheme above using standard in-order trace-collecting execution, we would obtain a trace that consisted only \Alice{} in her loop.
However, the projected program will have \Bob{} send 42 to \Cathy{} after some finite number of \Alice{}'s time around the loop.
The traces are no longer equivalent, and thus our proof technique fails.

In order to combat this, we plan to create a new in-order trace-collecting semantics which allows for programs to ``continue after an infinite loop.''
The traces collected by this semantics can have \emph{any} ordinal as their length, allowing us to make guarantees about \Bob{} and \Cathy{}'s projections in the program above.
This requires significant care to develop.
For instance, consider the choreography $\Alice.(\programfont{loop}\,()); \Alice.42 \ColSend \Bob$.
In this program, \Bob{} is waiting for a message from a process that is in an infinite loop, and therefore is equally ``blocked.''
We must keep track of such ``blocked'' processes throughout the semantics.

\begin{goal}
  \label{goal:strong-epp}
  Develop a new formalization of traces that allows us to establish a bisimulation in the presence of local infinite loops.
  Use it to develop strong guarantees for process-polymorphic code.
\end{goal}

\paragraph{Preliminary Work}
Both PIs have been working, along with a graduate student from each of Wisconsin and Buffalo, on developing the new paradigm of bisimulations.
In particular, we have developed a trace-collecting semantics that allows for ordinal-length traces to be collected, including reasoning about ``blocked'' processes as above.
We are currently trying to develop guarantees based on this semantics.

\subsubsection{Service Functions}
\label{sec:service-functions}

In order to provide a service, a process has to be willing to wait indefinitely to receive a request from an unknown client.
This puts the concept of services in contrast with the usual choreographic notion of deadlock freedom, which requires that either there is some step that a system can take or no process has pending communications.
We plan to address this by creating a notion of \emph{service functions} which act almost like normal polymorphic functions at the choreographic level.
For instance, we might have a service function with the following signature:
$$
\programfont{do\_service}[\programfont{providers}\; \Alice, \Bob](X : \programfont{client},\, x : X.\programfont{int})
$$
This declares a new service function \programfont{do\_service} provided by \Alice{} and \Bob{} with a client~$X$.
When projected to \Alice{} and \Bob{}, this will cause them to wait for some client~$X$ to request the service.
When the \programfont{do\_service} function is called in a choreography we discover who the client is.
For instance, we might have the call $\programfont{do\_service}\,(\Cathy, \Cathy.42)$, which tells us that \Cathy{} is the client of the function.
For \Cathy{}, this projects to a message send \Alice{} and \Bob{}, followed by her participation in the function.
For \Alice{} and \Bob{}, this projects to nothing: they are already waiting to participate in the function.

As mentioned above, one of the bigger wrinkles comes from the fact that services can become races: if two clients are trying to contact a service simultaneously, which request gets processed first is nondeterministic.
If we are able to build the new formalization of traces of Goal~\ref{goal:strong-epp}, then there is a simple way forward: when a provider starts and finishes a response to a client, they emit dual trace operations.
We then require traces to be well-bracketed---every start of a response to a client must be finished with an end of that same response---and consider traces to be equivalent if entire subtraces are commuted.
This complicates the theory traced above slightly, but retains its spirit.
If we do not successfully build this new formalization before developing the theory of service functions, then we believe that we can adapt previous arguments for deadlock freedom to this case.

In order for this to work, care must be taken.
First, we must make sure that the providers of a choreographic service do nothing but provide those services (perhaps after some initial setup).
If we allow them to participate in other code freely, then they may be attempting to communicate with a client when the client makes a request, leading to deadlock.
Similarly, any third party involved in responding to a request must be contacted through a service.
Otherwise, they will participate in the service once and then end.
When the service is requested again (perhaps by a different client), they will not be available to respond.
It is likely that other such restrictions will be required, as we shall discover when attempting to design such a system.

Once we have these special service functions, we can treat listening for a (collection of) service function(s) as a value, rather than as a pending message receive.
Since a service that is listening for more clients is not deadlocked, allowing this will not impact deadlock freedom.
This is similar to the approach taken to services in session types, which use replication to model services~\cite{DeYoungCPT09,CairesP10,Wadler12,LeBrunFD25}.
It is also similar to coalgebraic data types, which wait to be observed before reacting~\cite{KozenS17}.
We plan to use techniques from each to develop our theory of service functions.

\begin{goal}
  \label{goal:wait-until-called}
  Develop a theory of service functions, service providers, and clients.
  Prove that it is deadlock free.
\end{goal}


\iffalse
\akh{New story idea:
  We want services.
  This requires (a) new formulation of DF and (b) new techniques for proving DF from choreos and (c) new out-of-order stuff.
  We could do (b) and (c) in an ad-hoc way as we have before, but we think something more radical will work better: strong epp.
  Note, even if more radical thing fails, we can still do more ad-hoc stuff.
  If we don't get services, strong epp still makes everything else better.
}

Choreographies are designed to give a guarantee of deadlock freedom.
In order to do so, they give some form of connection between the semantics of the choreographic program itself and the semantics of the projections.
Traditionally, this has been in the form of a bisimulation~\cite{Montesi13,Cruz-FilipeM17,HirschG22,Montesi23}, a particularly strong connection.
However, PI~Hirsch showed in Pirouette that for functional choreographies a bisimulation result requires implicit synchronization between all of the processes in a choreogrpahy~\cite{HirschG22}.
Chor$\lambda$ gets around this by using commuting conversions in the semantics of choreographies, which allows choreographic code to move above code that may run after it once projected~\cite{CruzFilipeGLMP23,CruzFilipeGLMP22}.
However, the combined work of both PIs later showed that this solution breaks in the presence of named recursive functions, a vital feature of functional programming~\cite{SamuelsonHC25}.
Instead, they used a single simulation (rather than a \emph{bi}simulation) to argue for deadlock freedom; however, this means that other properties of the choreographic programming language cannot be cleanly lifted to the projected programs, as they can with a bisimulation argument.

To see what goes wrong, consider the program
$\Alice.\programfont{loop}; \Bob.42 \ColSend \Cathy$.
In this program, \Alice{} goes into an infinite loop while \Bob{} sends $42$ to \Cathy{}.
While \Alice{}'s instructions come before \Bob{}'s message to \Cathy{} in the choreography, because they involve disjoint sets of processes they are implicitly concurrent in the choreography.
A bisimulation would require that any action of the projected programs can take place in the choreography.
In particular, this means that \Bob{}'s message to \Cathy{} can occur even while \Alice{} is in her infinite loop (or, in fact, \emph{before} \Alice{} enters her infinite loop).
Traditionally \emph{out-of-order execution} allows this to happen in the choreographic semantics~\cite{Montesi23}, but this turns out to be impossible in the functional case~\cite{HirschG22,CruzFilipeGLMP23,CruzFilipeGLMP22}.
This lead the PIs to weakening the guarantee in order to establish deadlock freedom in the presence of process polymorphism~\cite{SamuelsonHC25}.

Once a bisimulation is established, the traditional argument goes as follows:
``The choreographic program will never ``get stuck,'' so either it has computed a value or it can take another step.
But if that's true of the choreography, then it must be true of the projected program as well, thanks to the bisimulation.
Therefore, the projected program can never be stuck with a cycle of processes waiting on each other, avoiding deadlock.''
This is a powerful argument.
However, it does not work in the presence of \emph{services} which always come back to wait for more input from an unknown client.
These are not traditional values, nor are they evidence of a deadlock.
Therefore, we must reformulate our notion of deadlock freedom as well as our proof techniques.

We propose to fix both these problems via (1) a new formalization of deadlock freedom and (2) a new paradigm for reestablishing bisimulations in the presence of infinite loops.
The new formalization of deadlock freedom will allow us to leverage process polymorphism to build choreographies in which some processes are services.
The new paradigm for reestablishing bisimulations will allow us to give strong guarantees to the projections of choreographic programs, even in the presence of process polymorphism.

\subsubsection{A New Formalization of Deadlock Freedom}
\label{sec:new-form-deadl}

Current formalizations of deadlock freedom say that no process is allowed to be waiting for a message.
However, a process that is providing a service will always be waiting for a message when it is not in the process of serving a request.
Thus, according to the standard formalization of deadlock freedom, no process that provides a service can be deadlock free.
This contradicts intuition and prevents choreographies from maintaining their goal of deadlock freedom by construction while allowing the development of services.
We plan to combat that contradiction by developing a new formalization of deadlock freedom.

We envision a choreographic service as a type of choreographic function in which some (concrete) processes \emph{provide} a service, while any other process(es) can act as (a) client(s).
The processes that provide a service wait to receive a \emph{request} from the clients, which causes the body of the function to run.
Importantly, while a service appears like any other function in the choreography, the processes providing the service do not get any information about when they are run.
Endpoint projection instead allows them to wait for a message signaling that some (statically unknown) client is requesting their service.
This message wait must be considered a value, rather than being considered a sign of deadlock, even when no client will ever again request the service.

This requires some care.
First, processes that provide a service must always eventually come to provide the same set of services (assuming that they are not in an infinite loop).
Otherwise, a process that is requesting a service may wait forever for the service to come back online, even though the provider has simply decided to take that service offline.
We plan to use ideas from coalgebraic data types and higher-order functions in order to ensure this.

Second, any third party---neither the provider nor the client of the service---mentioned in the body of the service function must also provide a service, and they may only be mentioned by calling their service.
Without this restriction, that third party may not be aware that the service is required, and cause deadlock.
We will use static analysis, likely embedded in the type system, to guarantee that this invariant is maintained.

Another wrinkle comes in the form of an ordering requirement.
Consider a choreography in which \Alice{} provides a service which is called twice, once by \Bob{} and once by \Cathy{} in that order.
After projection, this will appear as \Alice{} waiting for a request from an unknown client, while both \Bob{} and \Cathy{} attempt to make such a request.
This is a race, and so \Alice{} may start to process either request first.
Compare this situation to that of a standard choreographic function, which would force \Alice{} to enter the function with \Bob{} \emph{before} entering it with \Cathy{}.
We believe that this can be solved even with the fragile out-of-order evaluation procedures of previous work~\cite{CruzFilipeGLMP23,Montesi23,HirschG22}.
However, completing Goal~\ref{goal:strong-epp} will allow us to solve the problem more elegantly.

\begin{goal}
  \label{goal:wait-until-called}
  Develop a new formalization of deadlock freedom that allows for a wait-until-called paradigm.
  Use it to prove that a version of choreographies with services is deadlock free.
\end{goal}

\subsubsection{A New Paradigm for Bisimulations}
\label{sec:new-parad-bisim}

Recall that bisimulations require that any action taken by a choreographic program can be taken by its endpoint projection and vice-versa.
Since choreographies impose a textual ordering on actions between disjoint sets of processes, traditionally they allow for \emph{out-of-order execution} in their semantics in order to provide a bisimulation.
In the presence of named recursive functions this turns out to be impossible~\cite{SamuelsonHC25}, causing previous work to reach for weakening either the language or the guarantees~\cite{HirschG22,CruzFilipeGLMP22,SamuelsonHC25}.
We propose a new way to build bisimulations that does not require out-of-order execution and works in the presence of general named recursive functions.

Our key insight is that processes in a choreography always take the same actions in the same order, and actions can only be reordered between disjoint sets of processes.
If we look at a \emph{trace} of actions in a choreography, that trace will always contain the same actions in the same order as the projected code takes.
This ordering requirement will hold even if we arrange the actions of the trace, so long as we do not reorder any process's actions relative to each other.
These considerations lead to the following idea: run the choreography in-order, collecting a trace.
However, we consider that trace up to the equivalence generated by reordering the actions of disjoint processes.
We can then ``project'' those traces to a trace for every process; equivalent choreographic traces will project to equal traces for a given process.
A bisimulation then means that any choreographic trace corresponds to the traces of a possible run of the projected processes and vice-versa.
This is weaker in one sense to the more-traditional bisimulation, but because the trace contains all of the information about the semantics which we rely upon when making guarantees, that weakness is irrelevant.

The paradigm described above is straightforward for terminating programs.
Possible nontermination, however, leads to several wrinkles that must be considered.
First, if a process goes into an infinite loop in the choreography, no other events will be collected in the trace, even from other processes.
We plan to solve this by allowing for traces to be \emph{ordinal-length}, allowing for more trace elements to come ``after'' an infinite loop in the trace.
Second, if \Alice{} goes into an infinite loop, then \Bob{} waits for a message from \Alice, he will be stuck waiting for her.
Thus, no more of \Bob{}'s actions should appear in the trace.
We plan to solve this by keeping track of a set of ``blocked'' processes that grows over time
Not only do blocked processes take no actions in a choreography, but attempts to communicate with a blocked process will cause unblocked processes to become blocked.

Once we have this theory of traces and the services described above, we can easily describe the race condition considered above.
Simply put, when a service provider begins to process a request, they emit a trace event describing the clients of that request; once they finish processing the request, they emit a dual trace event.
We require that traces are well bracketed, and then consider traces equivalent if they rearrange entire bracketed subtraces.
This allows us to consider both the world in which \Alice{} serves \Bob{}'s request first and the world in which \Alice{} serves \Cathy{}'s request first equally.

\begin{goal}
  \label{goal:strong-epp}
  Develop a new formalization of traces that allows us to establish a bisimulation in the presence of local infinite loops.
  Use it to develop strong guarantees for process-polymorphic code.
\end{goal}
\fi


%%% Local Variables:
%%% mode: latex
%%% TeX-master: "desc"
%%% End:
