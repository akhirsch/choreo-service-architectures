%!TEX root = desc.tex

%thrust1
\subsection{Thrust 1: Polymorphism-Enabled Features}
\label{sec:t1}

The first thrust of this work will design new features for choreographic languages
that rely on first-class process polymorphism and
are needed to substantially broaden the range of possible choreographic applications.

\subsubsection{Spawning New Locations}

The first feature is the ability to spawn and kill new locations---computational units representing threads, processes, nodes, etc---% DO NOT DELETE, COMMENT IS IMPORTANT FOR SPACING
a core feature of many concurrent systems.
A few early choreographic works recognized its importance~\citep{CarboneM13,CruzFilipeM16a},
but those works lacked critical features for realistic programming, like higher-order operations,
and required locations to be spawned only to execute specific pre-defined operations.

We propose a much more general structure, where there is little distinction between an existing location and a newly-spawned one,
and threads may be killed when they are no longer needed.
Moreover, we will integrate this dynamic location management with a host of modern choreographic advances,
including higher-order functions and process polymorphism.

\ethan{There are multiple problems to consider here.
\begin{enumerate}[nosep]
  \item How does the new location know what code to execute?
  \item If you kill a location, how do you make sure you don't reference it again?
    This comes in two different forms: regular instructions, and killing it twice.
\end{enumerate}
We propose two ways of handling this: scoping and linearity.
We have preliminary work showing how to make scoping work, but it's missing some important features (e.g., thread pools).
We will finish that and figure out to make linearity work.}

%We propose two complementary approaches to supporting dynamic creation and termination of locations,
%both of which rely critically on location polymorphism.
%
%\paragraph{Preliminary Work: Scoped Threads}
%Preliminary work by both PIs and PI Cecchetti's student has made substantial progress
%towards one means

%Accomplishing this goal will make heavy use of location polymorphism.
%When a new location comes into existence, it is given a name
%that is then provided to an existing choreographic context.
%That context can then interact with the new location

%%% Local Variables:
%%% mode: latex
%%% TeX-master: "desc.tex"
%%% End:
