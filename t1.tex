%!TEX root = desc.tex

%thrust1
\subsection{Thrust 1: Polymorphism-Enabled Features}
\label{sec:t1}

The first thrust of this work will design new features for choreographic languages
that rely on first-class process polymorphism and
are needed to substantially broaden the range of possible choreographic applications.

\subsubsection{Spawning New Locations}
\label{sec:spawning}

The first feature is the ability to spawn and kill new locations---computational units representing threads, processes, nodes, etc---% DO NOT DELETE, COMMENT IS IMPORTANT FOR SPACING
a core feature of many concurrent systems.
A few early choreographic works recognized its importance~\citep{CarboneM13,CruzFilipeM16a},
but those works lacked critical features for realistic programming, like higher-order operations,
and required locations to be spawned only to execute specific pre-defined operations.

We propose a much more general structure, where there is little distinction between an existing location and a newly-spawned one,
and threads may be killed when they are no longer needed.
Moreover, we will integrate this dynamic location management with a host of modern choreographic advances,
including higher-order functions and process polymorphism.

\begin{example}{Dynamic Thread Pool}
Many systems incorporate thread pools that dynamically resize based on demand.
That is, at high-demand times when requests are regularly waiting on threads, they spawn more,
and at low-demand times when threads are regularly sitting idle, excess threads are killed.
The proposed work would allow us to implement such thread pools as choreographies
with code akin to the following.
\[
  \text{\ethan{Thread pool example code.}}
\]

The PIs recent work on first-class location polymorphism showed how to
implement \emph{static} thread pools with a fixed set of threads~\citep{SamuelsonHC25}.
However, that work assumes the set of threads in the entire system is fixed,
inherently preventing dynamically resizing the thread pool.
\end{example}

Safely spawning and killing locations in a choreography---be they threads, processes, or entire nodes---poses three distinct problems,
all illustrated by the above example.
\begin{enumerate}
  \item\label{spawn:li:name}
    Existing locations must be made aware of the new one in a way that allows them to interact with it.
    Otherwise the new location cannot send or receive messages, meaning they cannot participate in the choreography in any meaningful way.

    \textit{Example:} Clients of newly-spawned threads must know where to send their jobs.
  \item\label{spawn:li:code}
    The new location must know what code to execute, and it must match the code existing nodes expect.
    Without the correct code, the new location and existing locations might not have matching message sends and receives, causing deadlock.

    \textit{Example:} New threads must know to wait for instructions from a client and execute them
    and reply with the result, or the client will deadlock.
  \item\label{spawn:li:kill}
    If a location is killed, living locations must know not to send to or receive from the terminated location.
    A retained reference to a dead name can result in a live location attempting to contact a dead one, producing deadlock.

    \textit{Example:} The pool manager and clients must properly discard all references to a killed thread.
\end{enumerate}

The first concern is solved immediately by first-class location polymorphism.
Upon spawning a new thread, we give it a new name and bind that name to a location variable.
That location now functions just as any other polymorphic location variable;
it can be transmitted across the network or used as an endpoint for sending and receiving messages.

Problems~\ref{spawn:li:code} and~\ref{spawn:li:kill}---both of which can cause deadlock if not properly addressed---require additional research.
We propose two complementary strategies to addressing these concerns and retaining deadlock freedom with spawned locations:
scoped locations, and fractional permission tracking.

\paragraph{Preliminary Work: Scoped Locations}

\ethan{There are multiple problems to consider here.
\begin{enumerate}[nosep]
  \item How does the new location know what code to execute?
  \item If you kill a location, how do you make sure you don't reference it again?
    This comes in two different forms: regular instructions, and killing it twice.
\end{enumerate}
We propose two ways of handling this: scoping and linearity.
We have preliminary work showing how to make scoping work, but it's missing some important features (e.g., thread pools).
We will finish that and figure out to make linearity work.}

%We propose two complementary approaches to supporting dynamic creation and termination of locations,
%both of which rely critically on location polymorphism.
%
%\paragraph{Preliminary Work: Scoped Threads}
%Preliminary work by both PIs and PI Cecchetti's student has made substantial progress
%towards one means

%Accomplishing this goal will make heavy use of location polymorphism.
%When a new location comes into existence, it is given a name
%that is then provided to an existing choreographic context.
%That context can then interact with the new location

\subsection{Message Multi-Receives}
\label{sec:multi-receive}

\subsection{Events}
\label{sec:events}

\ethan{Andrew, I think I might need to leave this one to you.}

%%% Local Variables:
%%% mode: latex
%%% TeX-master: "desc.tex"
%%% End:
