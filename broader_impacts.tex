\section{Broader Impacts}
\label{sec:broader}

\paragraph{Societal Impacts}
Many concurrent and distributed systems are written in the form of services.
These include systems such a databases, HTTP servers, and thread pool managers.
Many of these systems are vital infrastructure for both the digital and physical worlds.
As such, we want to be able to rely on them; we therefore need to make significant guarantees about their properties.
Choreographies promise to enable the reasoning about concurrent systems necessary to make these guarantees; however, in order to fulfill that promise, we must empower process polymorphism as this grant proposes to do.

\paragraph{Broadening Participation}
Both PIs work regularly include undergraduates in their research teams.
PI~Hirsch participates in the University at Buffalo's \emph{Experiential Learning and Research~(ELR)} program, which gives undergraduate students the opportunity to experience research with a professor for several semesters.
PI~Cecchetti regularly encourages talented students in his courses to consider the opportunity.
Due to these efforts, multiple students have decided to pursue a graduate degree, and at least one student has decided to pursue functional programming professionally.
This grant will directly support these efforts.

PI~Hirsch also participates as an attending professional at Dragoncon, the world's largest fan-run convention.
This allows him to participate in science outreach with a passionate group of science communicators and reach an audience that struggles to learn about computer science.
In 2025, for instance, PI~Hirsch spoke on a panel about ``the science of the fiber arts,'' describing how computer science and arts such as knitting, weaving, and crocheting interplay.
Because of his participation in this activity, multiple members of the general public have reached out to PI~Hirsch about how to learn more about computer science.

\paragraph{Educational Impacts}
PI~Hirsch's undergraduate ELR group works on a choreographic-language compiler.
As part of this grant, PI~Cecchetti's undergraduate researchers will join the development team
and help extend that compiler with multiple implementations of process polymorphism and the various features based on process polymorphism developed by this grant (Goal~\ref{goal:opt-epp}).
Additionally, the undergraduates will participate in developing and implementing the process-polymorphism benchmark anticipated in Goal~\ref{goal:benchmark}.
As such, this grant will help train undergraduate students in functional programming, compiler construction, concurrency, and the scientific process.

Both PIs regularly teach courses for graduate students on the theory of programming languages which contain sections on concurrency.
We plan to add process polymorphism to these sections.
As such, this grant will also directly impact the education of graduate students in concurrency and programming-language theory.

\paragraph{Research Impacts}
We anticipate that this project will make significant impact in the further development of choreographic programming.
In particular, we anticipate that others will immediately adopt our new theory (Goals~\ref{goal:wait-until-called} and~\ref{goal:strong-epp}), changing the way people provide guarantees to choreographies.
We also anticipate that new attempts at practical choreographic languages will adopt our new primitives (Goals~\ref{goal:spawn} and~\ref{goal:multirecv}) and adapt our optimization of endpoint projection for process polymorphism (Goal~\ref{goal:opt-epp}).
Finally, we hope that the benchmark that we create for endpoint projection becomes a standard for the field as new process-polymorphic languages are developed and implemented (Goal~\ref{goal:benchmark}).

\paragraph{Dissemination}
The PIs will use standard dissemination best practices by publishing in conferences and journals.
To facilitate easy access, the software, formal models, proofs, educational material, and publications will be offered on the project website. 

%%% Local Variables:
%%% mode: latex
%%% TeX-master: "desc"
%%% End:
