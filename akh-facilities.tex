%!TEX root = desc.tex
\newif\ifdraft\drafttrue

\documentclass[11pt]{article}
\usepackage[margin=1in]{geometry}
\usepackage[sort&compress,numbers]{natbib}
% amsthm needs to go before newtx for compatability reasons
\usepackage{amsthm,amssymb}
\usepackage{newtx}
\usepackage{microtype}
\frenchspacing

\usepackage{mathpartir}
\usepackage{stmaryrd}
\usepackage{mathtools}

\usepackage{suffix}
\usepackage{xparse}

%\usepackage{pslatex}
%\usepackage{fullpage}
\usepackage{xspace}
\usepackage{verbatim}
\usepackage{enumitem}
%\usepackage{wrapfig}
%\usepackage{tgpagella}
\usepackage{url}
\usepackage{breakurl}
\usepackage[hidelinks,bookmarks=true]{hyperref}
\usepackage{graphicx}
\usepackage{subfig}
\usepackage{titlesec}
%\usepackage{fancyhdr}
\usepackage{float}
\usepackage[font=small]{caption}
\usepackage[table,dvipsnames]{xcolor}

\setlist[itemize,enumerate]{
  itemsep=0pt,
  topsep=1ex,
}

% ===== Title and section header formatting ====
\makeatletter
\def\@author{}
\def\author#1#2{\expandafter\def\expandafter\@author\expandafter{\@author{#1}{#2}}}
\def\typeset@authors#1#2#3\_nil{%
  \begin{tabular}[t]{c}#1\end{tabular}\\
  {\large #2}
  \ifx\relax#3\relax\else\and\typeset@authors#3\_nil\fi%
}
\def\maketitle{
  \newpage%
  \vbox to 1.2in{
    \centering
    \vspace*{\fill}
      {\LARGE\bfseries\@title\par}
      \vskip 0.75em
      {\Large\begin{tabular}[t]{c}\expandafter\typeset@authors\@author\_nil\end{tabular}\par}
    \vspace*{\fill}
  }
%  \begin{titlepage}
%    \centering
%    \vspace*{\fill}
%      {\huge\@title\par}
%      \vskip 2em
%      {\Large\begin{tabular}[t]{c}\expandafter\typeset@authors\@author\_nil\end{tabular}\par}
%    \vspace*{\fill}
%  \end{titlepage}
}
%\def\typeset@affils#1#2#3\_nil{\textsuperscript{#1}#2\ifx\relax#3\relax\else\and\typeset@affils#3\_nil\fi}
%\newcommand{\affils}[1]{\def\@affils{#1}}
%\def\maketitle{
%  \newpage%
%  \vbox to 0.9in{
%    \centering
%    \vspace*{\fill}
%      {\Large\bfseries\@title\par}
%      \vskip 1ex
%      {\normalsize\begin{tabular}[t]{c}\@author\end{tabular}\par}
%      \ifx\relax\@affils\relax\else
%        \vskip 0.5ex
%        {\small\begin{tabular}[t]{c}\expandafter\typeset@affils\@affils\_nil\end{tabular}\par}
%      \fi
%    \vspace*{\fill}
%  }
%}
\makeatother

\titlespacing*{\section}{0pt}{*2.5}{*2}
\titlespacing*{\subsection}{0pt}{*2}{*1}
%\titlespacing*{\subsubsection}{0pt}{*0.75}{*0.5}
\titleformat{\paragraph}[runin]{\bfseries}{}{0pt}{}[.]
\titlespacing{\paragraph}{0pt}{*1.5}{*1}
\titleformat{\subparagraph}[runin]{\bfseries}{}{0pt}{}
\titlespacing{\subparagraph}{0pt}{*1}{*1}

\makeatletter
\newcounter{ex@cntr}
\newenvironment{example}[1]{%
  \par\vspace{1.5ex plus 0.5ex minus 0.5ex}%
  \noindent%
  \addtocounter{ex@cntr}{1}%
  \textbf{Example \theex@cntr: #1.}\hspace{1ex plus 0.5ex minus 0.5ex}\ignorespaces%
}{\par\medskip}
\makeatother

%\renewcommand{\textfraction}{0.01}
%\setcounter{topnumber}{100}
%\setcounter{dbltopnumber}{100}
%\setcounter{totalnumber}{100}
%\renewcommand{\topfraction}{.9}
%\renewcommand{\floatpagefraction}{.9}
%\renewcommand{\dbltopfraction}{.9}
%\renewcommand{\dblfloatpagefraction}{.9}
%
%\newcounter{cnt}

%\setlength{\abovecaptionskip}{0in}
%\setlength{\belowcaptionskip}{0in}


%make latex layout figures better
%\renewcommand{\topfraction}{0.85}
%\renewcommand{\textfraction}{0.09}
%\renewcommand{\floatpagefraction}{0.85}


\date{}

%\newsavebox{\mylistingbox}
%
%\pagestyle{fancy}
%\fancyhead{}
%\fancyhf{}
%\renewcommand{\headrulewidth}{0pt}
%\setcounter{page}{1}
%%\usepackage{epstopdf}

% Notes
\newcommand{\newcommenter}[3]{%
  \ifdraft
    \newcommand{#1}[1]{%
      \begingroup\small\sffamily\color{#2}%
        [#3: ##1]%
      \endgroup%
    }%
  \else
    \newcommand{#1}[1]{\ignorespaces}
  \fi
}
\definecolor{darkgreen}{rgb}{0,0.7,0}

\newcommenter{\ethan}{darkgreen}{Ethan}
\newcommenter{\akh}{purple}{AKH}

\newcommand{\todo}[1]{\bgroup\small\sffamily\color{red}[\textbf{Todo:} #1]\egroup}

%% === Program Syntax ===
\newcounter{numlevels}
\NewDocumentCommand{\newMLP}{sO{l}mO{0}om}{
  \IfBooleanTF{#1}{
    \IfValueTF{#5}{
      \WithSuffix\newcommand#3*[#4][#5]
    }{
      \WithSuffix\newcommand#3*[#4]
    }
  }{
    \IfValueTF{#5}{
      \newcommand{#3}[#4][#5]
    }{
      \newcommand{#3}[#4]
    }
  }
  {\ifthenelse{\value{numlevels} > 0}{\begin{array}[t]{@{}#2@{}}}{\begin{array}{#2}}\addtocounter{numlevels}{1}#6\addtocounter{numlevels}{-1}\end{array}}
}
\makeatletter
\def\def@block@inner#1#2#3\_nil{#1 & #2\ifx\relax#3\relax\else \\ \def@block@inner#3\_nil\fi}%
\newcommand{\typeset@def@block}[2]{\begin{array}[t]{@{}l@{{}#1{}}l@{}}\def@block@inner#2\_nil\end{array}}
\makeatother

\newcommand{\ty}{\mkern2mu{:}\mkern2mu}
\newcommand{\proves}{\vdash}
\newcommand{\nproves}{\nvdash}
\newcommand{\dom}[1]{\operatorname{dom}({#1})}
\newcommand{\seq}{\mathrel{;}}
\newcommand{\sendsto}{\rightsquigarrow}

\definecolor{chorcolor}{HTML}{0053A9}
\definecolor{localcolor}{HTML}{007C21}
\definecolor{ntwkcolor}{HTML}{E44223}
\definecolor{choicecolor}{HTML}{BB7A00}
\definecolor{loccolor}{HTML}{b51963}
\newcommand{\ChorCol}[1]{{\color{chorcolor}#1}}
\newcommand{\LocalCol}[1]{{\color{localcolor}#1}}

\newcommand{\programfont}[1]{\ensuremath{\mathsf{#1}}\xspace}
\newcommand{\FontChoreo}[1]{\programfont{\ChorCol{#1}}}
\newcommand{\LocalLangFont}[1]{\programfont{\LocalCol{#1}}}
\newcommand{\LocFont}[1]{\programfont{\color{loccolor}#1}}

\newcommand{\Mngr}{\LocFont{M}}
\newcommand{\Alice}{\LocFont{A}}
\newcommand{\Bob}{\LocFont{B}}
\newcommand{\Client}{\LocFont{C}}

\newcommand{\ChorDef}{\mathrel{\ChorCol{\coloneqq}}}
\newcommand{\ColSend}{\mathrel{\ChorCol{\sendsto}}}

\newcommand{\IfN}{\FontChoreo{if}}
\newcommand{\Then}{\FontChoreo{then}}
\newcommand{\Else}{\FontChoreo{else}}
\newcommand{\ForkN}{\FontChoreo{fork}}
\newcommand{\LetN}{\FontChoreo{let}}
\newcommand{\In}{\FontChoreo{in}}
\newcommand{\FunN}{\FontChoreo{fun}}
\newcommand{\LamN}{\ChorCol{\lambda}}
\newcommand{\TLamN}{\ChorCol{\Lambda}}

\newcommand{\ForkIn}[3]{{#1}.\ForkN~{#2}~\In~{#3}}
\newMLP*{\ForkIn}[3]{{#1}.\ForkN~{#2} \\ \In~{#3}}
\newcommand{\LetIn}[3]{\LetN~{#1} \ChorDef {#2}~\In~{#3}}
\newMLP*[r@{~}l]{\LetIn}[3]{\LetN & {#1} \ChorDef {#2} \\ \In & {#3}}
\makeatletter
\newMLP{\LetMany}[2]{\LetN ~ {\typeset@def@block{\ChorDef}{#1}} \\ \In ~ {#2}}
\makeatother
\newcommand{\Fun}[3]{\operatorname{\FunN}{#1}({#2}) \ChorDef {#3}}
\newMLP*{\Fun}[3]{\operatorname{\FunN}{#1}({#2}) \ChorDef \\ {#3}}
\newcommand{\ITE}[3]{\IfN~{#1}~\Then~{#2}~\Else~{#3}}
\newMLP*{\ITE}[3]{\IfN~{#1} \\ \Then~{#2} \\ \Else~{#3}}

% === Outline structure ===
\usepackage{boxedminipage}
\makeatletter
\newcounter{outline@templateDepth}
\newcounter{outline@depth}
\newenvironment{outline}[2][0]{%
  \newlist{lvl}{enumerate}{6}%
  \setlist[lvl]{parsep=0.33\baselineskip,itemsep=0pt,topsep=0pt,leftmargin=1.5em}%
  \setlist[lvl,1]{leftmargin=*,topsep=0.5em}%
  %
  \setcounter{outline@templateDepth}{#1}%
  \setcounter{outline@depth}{1}%
  \@whilenum\value{outline@templateDepth}<6\do{%
    \ifcase\value{outline@templateDepth}\relax
      \setlist*[lvl,\value{outline@depth}]{label=\Roman*.}%
    \or
      \setlist*[lvl,\value{outline@depth}]{label=\Alph*.}%
    \or
      \setlist*[lvl,\value{outline@depth}]{label=\arabic*.}%
    \or
      \setlist*[lvl,\value{outline@depth}]{label=\alph*.}%
    \or
      \setlist*[lvl,\value{outline@depth}]{label=\roman*.}%
    \or
      \setlist*[lvl,\value{outline@depth}]{label=-}%
    \fi
    \stepcounter{outline@depth}%
    \stepcounter{outline@templateDepth}%
  }
  \if\relax\detokenize{#2}\relax\else
    \par\noindent{\bfseries #2}\par%
  \fi
  \noindent%
  \begin{lvl}%
}{\end{lvl}\vspace{0.25em}}
\newenvironment{boxedoutline}[2][0]{
  \begin{boxedminipage}{\textwidth}
  \begin{outline}[#1]{#2}
}{\end{outline}\end{boxedminipage}\vspace{0.5em}\par}
\makeatother

%%% Local Variables:
%%% mode: latex
%%% TeX-master: "desc"
%%% End:


\usepackage{paralist}

\begin{document}

\begin{center}
{\LARGE
\textsc{Facilities, Equipment, and Other Resources}
}
\end{center}
\hrule

\vspace{3mm}

The research activities will be conducted in the Computer Science and Engineering Department at the University at Buffalo (UB), the State University of New York.
The computational needs of the proposed research and educational activities are within the capabilities of the computing facilities in the department and university.

\subsection*{Programming Languages Lab}
PI~Hirsch has worked together with other faculty at UB to start a dedicated programming languages lab.
The lab consists of joint space for students as well as laptops and desktops for students and PI~Hirsch.
The lab holds a weekly seminar together with the database faculty, capitalizing the synergistic elements of both fields of study.

\subsection*{Department of Computer Science and Engineering Facilities}

Computer Science and Engineering (CSE) department maintains multiple information technology services and facilities to support its research mission.
These resources and facilities include (but are not limited to): 

\medskip 
\begin{compactitem}
  \item Storage infrastructures
  \item Compute services
  \item Lab and conference facilities
  \item Desktop infrastructures
  \item Application and database hosting
  \item Network and firewalling
  \item Disaster recovery
  \item Asset and license management / procurement
  \item Print and digital imaging services
  \item Security systems and environmental monitoring 
\end{compactitem}

\medskip \noindent
CSE research groups occupy over 6,600 square feet of research lab space ranging from secure, monitored, temperature-controlled data centers to specialized experimental facilities.
CSE instructional labs occupy over 4,000 square feet, configured to serve the characteristic needs of the courses they host.
In addition, CSE's four state-of-the-art data centers occupy over 2,100 square feet, all of which are environmentally conditioned and monitored 24/7. 

More than 270 Windows, MacOS and Linux PCs and thin client terminals are available across the multiple research, instructional and student labs.
Each lab is equipped with printing/imaging and presentation equipment.
Internet connectivity to all lab spaces is provided by 1 Gb/s Ethernet network connections.

CSE faculty, researchers and students also have access to compute labs administered by School of Engineering Node Services (SENS) and Computing and Information Technology (CIT).

\subsection*{Center for Computational Research (CCR) Facilities}

The Center for Computational Research (CCR) is a leading academic supercomputing facility that maintains a high-performance computing environment, high-end visualization laboratories, and support staff with expertise in computing, visualization, and networking to assist with computational needs of faculty and students.
The center's extensive computing facilities, which are housed in a state-of-the-art 4,000 square feet machine room, include a Linux cluster with more than 8,000 processor cores.
Its peak performance compute capacity exceeds 70 Tflops.
The center also maintains several high-performance storage systems of hundreds of TB, features a tiled display wall for projecting images in high resolution and a VisDuo passive stereo system for displaying complex 3D imagery. 

\subsection*{University Information Technologies Facilities}

In addition to the facilities described above, students and faculty have access to facilities and equipment owned and operated by UB Information Technology (UBIT).
Apart from UBIT running the CCR to support the university's research mission, it provides many other resources to the university community.
Some of the resources offered by UBIT include:

\medskip 
\begin{compactitem}
  \item Classroom and computer labs equipment;
  \item Voice, data, and video services such as phone, networking, video
    conferencing, wireless, etc.; 
  \item Training through classes, workshops, and documentation;
  \item Shared file space for use by students and faculty.
  \end{compactitem}

\end{document}

%%% Local Variables:
%%% mode: latex
%%% TeX-master: t
%%% End:
